\documentclass[12pt,a4paper]{article}
\usepackage{amsmath, amssymb, geometry}
\geometry{margin=1in}
\usepackage{siunitx}
\usepackage{bm}

\title{Week 01 Assignment Based on Book Computer Vision by Richard Szeliski, Second Edition, Chapter 2}
\author{Bishwash Khanal, \texttt{bishwash.b.khanal@student.jyu.fi}}
\date{September 13, 2025}

\begin{document}

\maketitle

\section{Pinhole Projection, Intrinsics, and Radial Distortion}

We are given a camera with resolution $1920\times 1080$ pixels and intrinsics
\[
K = \begin{bmatrix}
f_x & 0 & c_x\\
0 & f_y & c_y\\
0 & 0 & 1
\end{bmatrix}, 
\quad f_x=f_y=1200~\text{px}, \quad c_x=960, \quad c_y=540.
\]
The normalized pinhole model is
\[
x = \frac{X}{Z}, \qquad y = \frac{Y}{Z}, \qquad 
u = f_x x + c_x, \quad v = f_y y + c_y.
\]
Radial distortion is modeled as
\[
x_d = x\!\left(1+k_1 r^2+k_2 r^4\right), \quad
y_d = y\!\left(1+k_1 r^2+k_2 r^4\right),
\]
with $r^2=x^2+y^2$, $k_1=-0.25$, $k_2=0.10$.  
We consider the 3D point $(X,Y,Z)=(0.15,-0.05,2.0)$ meters.

\subsection*{(a) Compute $(u,v)$ ignoring distortion}

First compute normalized coordinates:
\[
x = \frac{0.15}{2.0} = 0.075, \qquad
y = \frac{-0.05}{2.0} = -0.025.
\]
Then apply intrinsic mapping:
\[
u = f_x x + c_x = 1200\cdot0.075 + 960 = 1050.0,
\]
\[
v = f_y y + c_y = 1200\cdot(-0.025) + 540 = 510.0.
\]
Thus,
\[
(u,v) = (1050.0,\ 510.0).
\]

\subsection*{(b) Compute $(u_d,v_d)$ with distortion}

Compute squared radius:
\[
r^2 = x^2 + y^2 = (0.075)^2 + (-0.025)^2 = 0.00625.
\]
Radial factor:
\[
\alpha = 1 + k_1 r^2 + k_2 r^4 
= 1 - 0.25\cdot0.00625 + 0.10\cdot(0.00625^2).
\]
Numerically,
\[
\alpha \approx 0.99844.
\]
Apply distortion:
\[
x_d = x\alpha \approx 0.075 \cdot 0.99844 = 0.07488, \qquad
y_d = y\alpha \approx -0.025 \cdot 0.99844 = -0.02496.
\]
Convert to pixel coordinates:
\[
u_d = f_x x_d + c_x = 1200\cdot0.07488 + 960 \approx 1049.86,
\]
\[
v_d = f_y y_d + c_y = 1200\cdot(-0.02496) + 540 \approx 510.05.
\]
Thus,
\[
(u_d,v_d) \approx (1049.86,\ 510.05).
\]

\subsection*{(c) Fields of View}

The horizontal field of view is
\[
\mathrm{FOV}_x = 2 \arctan\!\left(\frac{W}{2f_x}\right)
= 2 \arctan\!\left(\frac{1920}{2400}\right).
\]
Numerically,
\[
\mathrm{FOV}_x \approx 77.3^\circ.
\]

Similarly, the vertical field of view:
\[
\mathrm{FOV}_y = 2 \arctan\!\left(\frac{H}{2f_y}\right)
= 2 \arctan\!\left(\frac{1080}{2400}\right).
\]
Numerically,
\[
\mathrm{FOV}_y \approx 48.5^\circ.
\]

\subsection*{(d) Effective focal length in mm}

Pixel pitch is given as $p=4~\mu\mathrm{m} = 0.004~\mathrm{mm}$.  
Effective focal length:
\[
f_{\mathrm{mm}} = f_x \cdot p = 1200 \cdot 0.004 = 4.8~\mathrm{mm}.
\]

\subsection*{(e) Pixel shift in $u$ due to distortion}

The horizontal shift is
\[
\Delta u = u_d - u = 1049.86 - 1050.0 \approx -0.14\ \text{px}.
\]
Thus, the distortion shifted the point leftward by about $0.14$ pixels.

\section*{Final Results}
\begin{itemize}
\item Ignoring distortion: $(u,v)=(1050.0,\ 510.0)$.
\item With distortion: $(u_d,v_d)\approx(1049.86,\ 510.05)$.
\item FOVs: $\mathrm{FOV}_x\approx 77.3^\circ$, $\mathrm{FOV}_y\approx 48.5^\circ$.
\item Effective focal length: $f_{\mathrm{mm}}=4.8$ mm.
\item Pixel shift: $\Delta u \approx -0.14$ px.
\end{itemize}

% -----------------------------------------------------------------------------

% -----------------------------------------------------------------------------

\section{Thin Lens, Depth of Field, and Diffraction}

\textbf{Given:} focal length \(f=\SI{35}{mm}\), pixel pitch \(p=\SI{4}{\micro m}\), wavelength \(\lambda=\SI{550}{nm}\),
circle of confusion \(c=2p\), subject distance \(s=\SI{2.5}{m}\).  
(Throughout we keep units explicit and convert where necessary.)

\bigskip

\subsection*{(a) Hy\-per\-fo\-cal distance at \(N=4\)}

The hyperfocal distance for a given circle of confusion \(c\) and f-number \(N\) is
\[
H=\frac{f^2}{N c}+f.
\]
We must use consistent units. Convert the circle of confusion to \(\mathrm{mm}\):
\[
p=\SI{4}{\micro m}=\SI{0.004}{mm}\quad\Rightarrow\quad c=2p=\SI{0.008}{mm}.
\]
Compute \(f^2\) (in \(\text{mm}^2\)): \(f^2=(\SI{35}{mm})^2=\SI{1225}{mm^2}\).
Then
\[
\frac{f^2}{N c}=\frac{1225}{4\cdot 0.008}=\frac{1225}{0.032}=\SI{38281.25}{mm}.
\]
Add \(f\) (small compared to the fraction):
\[
H=\SI{38281.25}{mm}+\SI{35}{mm}=\SI{38316.25}{mm}=\boxed{\SI{38.316}{m}}.
\]
The hyperfocal distance is very large (tens of metres) because the chosen circle of confusion is tiny (8 $\mu$m).

\bigskip

\subsection*{(b) Near and far depth-of-field limits when focused at \(s=\SI{2.5}{m}\)}

Use the standard formulas (with all distances in the same units; here \(\mathrm{mm}\)):
\[
s_{\text{near}}=\frac{H s}{H+(s-f)},\qquad
s_{\text{far}}=\frac{H s}{H-(s-f)}.
\]
Convert \(s\) and \(f\) to \(\mathrm{mm}\):
\[
s=\SI{2.5}{m}=\SI{2500}{mm},\qquad f=\SI{35}{mm}.
\]
Compute \(s-f=\SI{2500}{mm}-\SI{35}{mm}=\SI{2465}{mm}\). Using \(H=\SI{38316.25}{mm}\) from (a):
\[
s_{\text{near}}=\frac{38316.25\cdot 2500}{38316.25+2465}
=\frac{95\,790\,625}{40\,781.25}\approx\SI{2348.89}{mm}=\boxed{\SI{2.349}{m}}.
\]
\[
s_{\text{far}}=\frac{38316.25\cdot 2500}{38316.25-2465}
=\frac{95\,790\,625}{35\,851.25}\approx\SI{2671.89}{mm}=\boxed{\SI{2.672}{m}}.
\]
Since \(s<H\) holds, both near and far distances are finite. The DoF ranges approximately from \(\SI{2.349}{m}\) to \(\SI{2.672}{m}\).

\bigskip

\subsection*{(c) Airy disk diameter at \(N=16\)}

The diameter of the diffraction-limited Airy disk (first zero to first zero) on the image plane is
\[
d_{\text{Airy}} = 2.44\,\lambda\,N.
\]
Convert the wavelength to micrometres for convenience:
\[
\lambda=\SI{550}{nm}=\SI{0.55}{\micro m}.
\]
At \(N=16\):
\[
d_{\text{Airy}} = 2.44\cdot 0.55\cdot 16
= (2.44\cdot 0.55)\cdot 16 = 1.342\cdot 16 = \SI{21.472}{\micro m}.
\]
Expressed in pixels (pixel pitch \(p=\SI{4}{\micro m}\)):
\[
\frac{d_{\text{Airy}}}{p}=\frac{21.472}{4}\approx\boxed{5.368\ \text{pixels}}.
\]
At \(N=16\) the Airy diameter spans more than 5 pixels, so diffraction blur is much larger than a single pixel.

\bigskip

\subsection*{(d) f-number \(N^\ast\) where \(d_{\text{Airy}}=2\) pixels (i.e.\ \(=\SI{8}{\micro m}\))}

We set \(d_{\text{Airy}}=2.44\,\lambda\,N^\ast = \SI{8}{\micro m}\). Solve for \(N^\ast\):
\[
N^\ast = \frac{8}{2.44\,\lambda(\SI{}{\micro m})} = \frac{8}{2.44\cdot 0.55}.
\]
Compute denominator: \(2.44\cdot 0.55 = 1.342\). Hence
\[
N^\ast = \frac{8}{1.342}\approx \boxed{5.96}.
\]
Rounding reasonably, \(N^\ast\approx 6\).

\textbf{Is the system diffraction- or sensor-limited at \(N^\ast\)?}  
At \(N^\ast\) the Airy disk diameter equals \(\SI{8}{\micro m}\), i.e.\ two pixels (each \(\SI{4}{\micro m}\)). This means diffraction produces a blur larger than a single pixel: the point spread is dominated by diffraction rather than by sensor sampling. Therefore the system is \emph{diffraction-limited} (or at the diffraction–sampling transition). For \(N>N^\ast\) diffraction will increasingly dominate; for \(N<N^\ast\) the sensor (pixel size / optical aberrations) is more likely to limit resolving power.

\bigskip

\section*{Final Results}
\begin{itemize}
  \item \(H=\SI{38316.25}{mm}=\SI{38.316}{m}.\)
  \item \(s_{\text{near}}\approx\SI{2348.89}{mm}=\SI{2.349}{m}.\)
  \item \(s_{\text{far}}\approx\SI{2671.89}{mm}=\SI{2.672}{m}.\)
  \item \(d_{\text{Airy}}(N=16)=\SI{21.472}{\micro m}\approx 5.37\ \text{pixels}.\)
  \item \(N^\ast\approx 5.96\) (approx.\ \(6\)); at this f-number the system is diffraction-limited.
\end{itemize}

% -----------------------------------------------------------------------------

% -----------------------------------------------------------------------------

\section{Sampling, Nyquist, and Aliasing}

\textbf{Given:} pixel pitch \(p=\SI{4}{\micro m}=\SI{0.004}{mm}\).  
Sampling frequency at the sensor plane is defined as
\[
f_s=\frac{1}{p}\quad\text{(cycles/mm)},\qquad f_N=\frac{f_s}{2}.
\]
Three sinusoidal gratings at the sensor plane have spatial frequencies
\[
\text{A: } f_A=\SI{90}{cycles/mm},\qquad
\text{B: } f_B=\SI{150}{cycles/mm},\qquad
\text{C: } f_C=\SI{210}{cycles/mm}.
\]

\subsection*{(1) Compute \(f_s\) and \(f_N\). Which gratings exceed Nyquist?}

Compute the sampling frequency (use \(p=\SI{0.004}{mm}\)):
\[
f_s=\frac{1}{p}=\frac{1}{0.004}=\SI{250}{cycles/mm}.
\]
Thus the Nyquist frequency is
\[
f_N=\frac{f_s}{2}=\frac{250}{2}=\SI{125}{cycles/mm}.
\]

Compare each grating with \(f_N=\SI{125}{cycles/mm}\):
\begin{itemize}
  \item A: \(90 < 125\) — below Nyquist (no aliasing).
  \item B: \(150 > 125\) — exceeds Nyquist (will alias).
  \item C: \(210 > 125\) — exceeds Nyquist (will alias).
\end{itemize}

\subsection*{(2) Aliased frequencies for those exceeding Nyquist}

We compute aliased frequency using
\[
f_{\text{alias}}=\left|\,f - \operatorname{round}\!\left(\frac{f}{f_s}\right)f_s\,\right|.
\]
Here \(f_s=\SI{250}{cycles/mm}\).

For \(f_B=\SI{150}{cycles/mm}\):
\[
\frac{f_B}{f_s}=\frac{150}{250}=0.6 \quad\Rightarrow\quad \operatorname{round}(0.6)=1,
\]
so
\[
f_{B,\text{alias}}=\big|150 - 1\cdot250\big|=\big|-100\big|=\SI{100}{cycles/mm}.
\]

For \(f_C=\SI{210}{cycles/mm}\):
\[
\frac{f_C}{f_s}=\frac{210}{250}=0.84 \quad\Rightarrow\quad \operatorname{round}(0.84)=1,
\]
so
\[
f_{C,\text{alias}}=\big|210 - 1\cdot250\big|=\big|-40\big|=\SI{40}{cycles/mm}.
\]

(If a frequency lay above \(1.5f_s\) the rounding could give 2, etc.; here both round to 1.)

\subsection*{(3) Convert each grating’s (true or aliased) frequency to cycles/pixel and period in pixels}

Relation:
\[
\text{cycles/pixel} = f\ [\text{cycles/mm}]\times p\ [\text{mm/pixel}],\qquad
\text{period (pixels)} = \frac{1}{\text{cycles/pixel}}.
\]
With \(p=\SI{0.004}{mm/pixel}\):

\paragraph{A (no alias):}
\[
\text{cycles/pixel}_A = 90\cdot0.004 = 0.36\ \text{cycles/pixel},
\qquad \text{period}_A = \frac{1}{0.36}\approx 2.7778\ \text{pixels}.
\]

\paragraph{B (aliased to \(\SI{100}{cycles/mm}\)):}
\[
\text{cycles/pixel}_B = 100\cdot0.004 = 0.40\ \text{cycles/pixel},
\qquad \text{period}_B = \frac{1}{0.40} = 2.5\ \text{pixels}.
\]

\paragraph{C (aliased to \(\SI{40}{cycles/mm}\)):}
\[
\text{cycles/pixel}_C = 40\cdot0.004 = 0.16\ \text{cycles/pixel},
\qquad \text{period}_C = \frac{1}{0.16} = 6.25\ \text{pixels}.
\]

\subsection*{(4) Downsample by factor 2 after ideal low-pass filtering: maximum passband cutoff (cycles/pixel) to guarantee no aliasing}

When downsampling by an integer factor \(D\) the new sampling frequency becomes \(f_s' = f_s/D\), and the new Nyquist (in cycles/mm) is \(f_N' = f_s' / 2 = f_s/(2D)\). Converting to cycles/pixel (multiply by \(p\)) gives
\[
\text{cutoff} = f_N' \cdot p = \frac{f_s}{2D}\,p.
\]
But \(f_s p = (1/p)\cdot p = 1\), so this simplifies to
\[
\boxed{\text{cutoff}=\frac{1}{2D}\ \text{cycles/pixel}.}
\]
For \(D=2\):
\[
\text{cutoff}=\frac{1}{4}=\boxed{0.25\ \text{cycles/pixel}}.
\]

\subsection*{(5) Downsample by factor 3: required cutoff}

For \(D=3\):
\[
\text{cutoff}=\frac{1}{2D}=\frac{1}{6}\approx\boxed{0.1667\ \text{cycles/pixel}}.
\]
(Equivalently: in cycles/mm this is \(f_N' = f_s/(2\cdot 3)=250/6\approx\SI{41.6667}{cycles/mm}\); multiplying by \(p=\SI{0.004}{mm}\) yields \(0.1667\) cycles/pixel.)

\bigskip
\section*{Final Results}
\begin{itemize}
  \item \(f_s=\SI{250}{cycles/mm}\), \(f_N=\SI{125}{cycles/mm}\).
  \item A (\(\SI{90}{cycles/mm}\)) is below Nyquist. B (\(\SI{150}{cycles/mm}\)) and C (\(\SI{210}{cycles/mm}\)) exceed Nyquist.
  \item Aliases: \(f_{B,\text{alias}}=\SI{100}{cycles/mm}\), \(f_{C,\text{alias}}=\SI{40}{cycles/mm}\).
  \item Cycles/pixel and periods:
    \begin{itemize}
      \item A: \(0.36\ \text{cycles/pixel}\), period \(\approx 2.78\) px.
      \item B (aliased): \(0.40\ \text{cycles/pixel}\), period \(2.50\) px.
      \item C (aliased): \(0.16\ \text{cycles/pixel}\), period \(6.25\) px.
    \end{itemize}
  \item Downsample factor 2: cutoff \(=0.25\) cycles/pixel. \\
        Downsample factor 3: cutoff \(\approx 0.1667\) cycles/pixel.
\end{itemize}

\end{document}
