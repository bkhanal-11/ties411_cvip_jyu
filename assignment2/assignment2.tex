\documentclass[12pt,a4paper]{article}
\usepackage{amsmath, amssymb, geometry}
\geometry{margin=1in}
\usepackage{siunitx}
\usepackage{bm}
\usepackage{url}

\title{Week 02 Assignment Based on Book Computer Vision by Richard Szeliski, Second Edition, Chapter 3}
\author{Bishwash Khanal, \texttt{bishwash.b.khanal@student.jyu.fi}}
\date{September 21, 2025}

\begin{document}

\maketitle

\section*{3.3 Compositing and reflections.}
Section 3.1.3 describes the process of compositing an alpha-matted image on top of another. Answer the following questions and optionally
validate them experimentally:
\begin{enumerate}
    \item Most captured images have gamma correction applied to them. Does this invalidate the
basic compositing equation (3.8); if so, how should it be fixed?
    

$\Rightarrow$ Section 3.1.3 describes the process of compositing an alpha-matted image on top of another. The basic compositing equation is:
\begin{equation}
C = (1 - \alpha)B + \alpha F
\end{equation}
where $C$ is the composite image, $B$ is the background, $F$ is the foreground, and $\alpha$ is the alpha matte.

Yes, gamma correction does invalidate the basic compositing equation. The compositing equation assumes \emph{linear blending} of light intensities. 
However, most digital images undergo gamma correction during capture:
\begin{equation}
I_{display} = I_{linear}^{1/\gamma}
\end{equation}
where $\gamma \approx 2.2$ for most display systems.

When we apply the compositing equation directly to gamma-corrected values, we get:
\begin{equation}
C_{gamma} = (1 - \alpha)B_{gamma} + \alpha F_{gamma}
\end{equation}

This is \emph{not} equivalent to the physically correct linear blending followed by gamma correction.

To fix this, we must:

\begin{enumerate}
\item \textbf{Linearize} the input images:
\begin{align}
B_{linear} &= B_{gamma}^{\gamma}\\
F_{linear} &= F_{gamma}^{\gamma}
\end{align}

\item \textbf{Composite} in linear space:
\begin{equation}
C_{linear} = (1 - \alpha)B_{linear} + \alpha F_{linear}
\end{equation}

\item \textbf{Apply gamma correction} to the result:
\begin{equation}
C_{gamma} = C_{linear}^{1/\gamma}
\end{equation}
\end{enumerate}

Let's show that direct gamma compositing $\neq$ correct linear compositing:

Direct gamma compositing:
\begin{equation}
C_1 = (1-\alpha)B_{gamma} + \alpha F_{gamma} = (1-\alpha)B_{linear}^{1/\gamma} + \alpha F_{linear}^{1/\gamma}
\end{equation}

Correct linear compositing:
\begin{equation}
C_2 = [(1-\alpha)B_{linear} + \alpha F_{linear}]^{1/\gamma}
\end{equation}

Since the gamma function is nonlinear, $C_1 \neq C_2$ in general.

\item The additive (pure reflection) model may have limitations. What happens if the glass is
tinted, especially to a non-gray hue? How about if the glass is dirty or smudged? How
could you model wavy glass or other kinds of refractive objects?
\end{enumerate}

$\Rightarrow$ The additive model assumes: $C = T + R$ where $T$ is transmitted light and $R$ is reflected light.

\textbf{Tinted glass:} A tinted glass acts as a colored filter, multiplying transmitted light by wavelength-dependent coefficients. 
The model becomes $C = (1 - \alpha)B + \alpha(T \odot F + R \odot \text{reflection})$ where T and R are the transmission 
and reflection color coefficients, and $\odot$ denotes element-wise multiplication.

\textbf{Dirty/smudged glass:} This introduces scattering and diffusion effects. Scattering can be modeled by convolving the reflection 
with a blur kernel. Dirt particles create spatially varying transmission coefficients. The model can be expressed as: 
$C = (1 - \alpha)B + \alpha$(T(x,y) $\cdot$ F + blur(R(x,y) $\cdot$ reflection)).

\textbf{Wavy/refractive objects:} These cause geometric distortions requiring warping operations. To model these effects, 
we need to: model the surface normal variations, apply appropriate refraction using Snell's law to warp the transmitted image, 
and account for spatially varying magnification and distortion.

\section*{3.11 Discrete Gaussian filters.}
Discuss the following issues with implementing a discrete Gaussian filter:
\begin{itemize}
    \item If you just sample the equation of a continuous Gaussian filter at discrete locations,
will you get the desired properties, e.g., will the coefficients sum up to 1? Similarly, if
you sample a derivative of a Gaussian, do the samples sum up to 0 or have vanishing
higher-order moments?

$\Rightarrow$ Simply sampling a continuous Gaussian at discrete points will NOT generally sum to 1. The sum 
depends on the sampling rate and filter width. You must normalize: $h_{normalized}[i] = h[i] / \Sigma h[i]$. Sampled derivatives 
of Gaussians also won't have exactly zero sum or correct higher-order moments due to discretization errors.

    \item Would it be preferable to take the original signal, interpolate it with a sinc, blur with a
continuous Gaussian, then prefilter with a sinc before re-sampling? Is there a simpler
way to do this in the frequency domain?

$\Rightarrow$ Sinc interpolation approach is theoretically optimal but computationally expensive. This involves 
interpolating with sinc to create a continuous signal, convolving with continuous Gaussian, applying an anti-alias filter with 
sinc, and then resampling. In the frequency domain, it's much simpler - multiply FFT of signal by Gaussian frequency response, 
then take IFFT.

    \item Would it make more sense to produce a Gaussian frequency response in the Fourier
domain and to then take an inverse FFT to obtain a discrete filter?

$\Rightarrow$ Direct frequency design often produces better results. Define desired Gaussian frequency response 
$H(\omega) = e^{-\omega^2\sigma^2/2}$, take inverse FFT to get spatial filter, then truncate and shift appropriately.

    \item How does truncation of the filter change its frequency response? Does it introduce any
additional artifacts?

$\Rightarrow$ Truncation introduces ripples (Gibbs phenomenon) in frequency response. The abrupt cutoff creates 
high-frequency artifacts. Use smooth windowing (Hamming, Hann) instead of rectangular truncation to reduce artifacts.
    \item Are the resulting two-dimensional filters as rotationally invariant as their continuous
analogs? Is there some way to improve this? In fact, can any 2D discrete (separable or
non-separable) filter be truly rotationally invariant?

$\Rightarrow$ 2D discrete filters cannot be perfectly rotationally invariant due to the discrete grid. The best 
you can achieve is approximate invariance. Improvements include using larger support regions, designing filters that 
minimize directional bias, using steerable filters, and employing non-separable filters which generally perform better 
than separable ones for rotational invariance. Theoretically, no discrete filter can be truly rotationally invariant - 
the discrete grid imposes inherent anisotropy.
\end{itemize}


\section*{3.15 Fourier transform.}
Prove the properties of the Fourier transform listed in Szeliski (2010, Table 3.1) and derive the formulas for the Fourier 
transforms pairs listed in Szeliski (2010, Table 3.2) and Table 3.1. These exercises are very useful if you want to become 
comfortable working with Fourier transforms, which is a very useful skill when analyzing and designing the behavior and 
efficiency of many computer vision algorithms.

\subsection*{Properties from Table 3.1}

\subsubsection*{1. Superposition}
\textbf{Property}: $f_1(x) + f_2(x) \Leftrightarrow F_1(\omega) + F_2(\omega)$
\newline
\textbf{Proof}:
\begin{align*}
\mathcal{F}\{f_1(x) + f_2(x)\} &= \int_{-\infty}^{\infty} [f_1(x) + f_2(x)]e^{-j\omega x}dx\\
&= \int_{-\infty}^{\infty} f_1(x)e^{-j\omega x}dx + \int_{-\infty}^{\infty} f_2(x)e^{-j\omega x}dx\\
&= F_1(\omega) + F_2(\omega)
\end{align*}

\subsubsection*{2. Shift}
\textbf{Property}: $f(x - x_0) \Leftrightarrow F(\omega)e^{-j\omega x_0}$
\newline
\textbf{Proof}: Using substitution $u = x - x_0$:
\begin{align*}
\mathcal{F}\{f(x-x_0)\} &= \int_{-\infty}^{\infty} f(x-x_0)e^{-j\omega x}dx\\
&= \int_{-\infty}^{\infty} f(u)e^{-j\omega (u+x_0)}du\\
&= e^{-j\omega x_0}\int_{-\infty}^{\infty} f(u)e^{-j\omega u}du\\
&= e^{-j\omega x_0}F(\omega)
\end{align*}

\subsubsection*{3. Reversal}
\textbf{Property}: $f(-x) \Leftrightarrow F^*(-\omega)$
\newline
\textbf{Proof}: Using substitution $u = -x$:
\begin{align*}
\mathcal{F}\{f(-x)\} &= \int_{-\infty}^{\infty} f(-x)e^{-j\omega x}dx\\
&= \int_{\infty}^{-\infty} f(u)e^{-j\omega (-u)}(-du)\\
&= \int_{-\infty}^{\infty} f(u)e^{j\omega u}du\\
&= \left[\int_{-\infty}^{\infty} f(u)e^{-j(-\omega) u}du\right]^*\\
&= F^*(-\omega)
\end{align*}

\subsubsection*{4. Convolution}
\textbf{Property}: $f(x) * h(x) \Leftrightarrow F(\omega)H(\omega)$
\newline
\textbf{Proof}:
\begin{align*}
\mathcal{F}\{f * h\} &= \mathcal{F}\left\{\int_{-\infty}^{\infty} f(\tau)h(x-\tau)d\tau\right\}\\
&= \int_{-\infty}^{\infty} \left[\int_{-\infty}^{\infty} f(\tau)h(x-\tau)d\tau\right]e^{-j\omega x}dx\\
&= \int_{-\infty}^{\infty} f(\tau)\left[\int_{-\infty}^{\infty} h(x-\tau)e^{-j\omega x}dx\right]d\tau\\
&= \int_{-\infty}^{\infty} f(\tau)H(\omega)e^{-j\omega \tau}d\tau \quad \text{(using shift property)}\\
&= H(\omega)\int_{-\infty}^{\infty} f(\tau)e^{-j\omega \tau}d\tau\\
&= F(\omega)H(\omega)
\end{align*}

\subsubsection*{5. Correlation}
\textbf{Property}: $f(x) \otimes h(x) \Leftrightarrow F(\omega)H^*(\omega)$
\newline
\textbf{Note}: Correlation is defined as $f \otimes h = \int_{-\infty}^{\infty} f(\tau)h(\tau + x)d\tau$
\newline
\textbf{Proof}:
\begin{align*}
\mathcal{F}\{f \otimes h\} &= \mathcal{F}\left\{\int_{-\infty}^{\infty} f(\tau)h(\tau + x)d\tau\right\}\\
&= \int_{-\infty}^{\infty} f(\tau)\mathcal{F}\{h(\tau + x)\}d\tau\\
&= \int_{-\infty}^{\infty} f(\tau)H(\omega)e^{j\omega \tau}d\tau \quad \text{(shift property)}\\
&= H(\omega)\int_{-\infty}^{\infty} f(\tau)[e^{-j\omega \tau}]^*d\tau\\
&= H(\omega)[F(\omega)]^* = F(\omega)H^*(\omega)
\end{align*}

\subsubsection*{6. Multiplication}
\textbf{Property}: $f(x)h(x) \Leftrightarrow F(\omega) * H(\omega)$
\newline
\textbf{Proof}: Using the inverse Fourier transform (following Szeliski's convention):
\begin{align*}
f(x) &= \int_{-\infty}^{\infty} F(\alpha)e^{j\alpha x}d\alpha\\
f(x)h(x) &= h(x) \int_{-\infty}^{\infty} F(\alpha)e^{j\alpha x}d\alpha\\
\mathcal{F}\{f(x)h(x)\} &= \int_{-\infty}^{\infty} F(\alpha)\mathcal{F}\{h(x)e^{j\alpha x}\}d\alpha\\
&= \int_{-\infty}^{\infty} F(\alpha)H(\omega - \alpha)d\alpha\\
&= F(\omega) * H(\omega)
\end{align*}

\subsubsection*{7. Differentiation}
\textbf{Property}: $f'(x) \Leftrightarrow j\omega F(\omega)$
\newline
\textbf{Proof}: Using integration by parts:
\begin{align*}
\mathcal{F}\{f'(x)\} &= \int_{-\infty}^{\infty} f'(x)e^{-j\omega x}dx\\
&= [f(x)e^{-j\omega x}]_{-\infty}^{\infty} - \int_{-\infty}^{\infty} f(x)(-j\omega)e^{-j\omega x}dx\\
&= 0 + j\omega\int_{-\infty}^{\infty} f(x)e^{-j\omega x}dx\\
&= j\omega F(\omega)
\end{align*}

\subsubsection*{8. Domain Scaling}
\textbf{Property}: $f(ax) \Leftrightarrow \frac{1}{a}F\left(\frac{\omega}{a}\right)$ (for $a > 0$)
\newline
\textbf{Proof}: Using substitution $u = ax$:
\begin{align*}
\mathcal{F}\{f(ax)\} &= \int_{-\infty}^{\infty} f(ax)e^{-j\omega x}dx\\
&= \int_{-\infty}^{\infty} f(u)e^{-j\omega u/a}\frac{du}{a}\\
&= \frac{1}{a}\int_{-\infty}^{\infty} f(u)e^{-j(\omega/a) u}du\\
&= \frac{1}{a}F\left(\frac{\omega}{a}\right)
\end{align*}

\subsubsection*{9. Real Images}
\textbf{Property}: $f(x) = f^*(x) \Leftrightarrow F(\omega) = F^*(-\omega)$
\newline
\textbf{Proof}: For real-valued $f(x)$:
\begin{align*}
F^*(\omega) &= \left[\int_{-\infty}^{\infty} f(x)e^{-j\omega x}dx\right]^*\\
&= \int_{-\infty}^{\infty} f^*(x)e^{j\omega x}dx\\
&= \int_{-\infty}^{\infty} f(x)e^{j\omega x}dx \quad \text{(since $f$ is real)}\\
&= \int_{-\infty}^{\infty} f(x)e^{-j(-\omega) x}dx\\
&= F(-\omega)
\end{align*}

\subsubsection*{10. Parseval's Theorem}
\textbf{Property}: $\sum_x [f(x)]^2 = \sum_\omega [F(\omega)]^2$
\newline
For continuous signals: $\int_{-\infty}^{\infty} |f(x)|^2 dx = \int_{-\infty}^{\infty} |F(\omega)|^2 d\omega$
\newline
\textbf{Proof}:
\begin{align*}
\int_{-\infty}^{\infty} |f(x)|^2 dx &= \int_{-\infty}^{\infty} f(x)f^*(x) dx\\
&= \int_{-\infty}^{\infty} f(x)\left[\int_{-\infty}^{\infty} F^*(\omega)e^{-j\omega x}d\omega\right] dx\\
&= \int_{-\infty}^{\infty} F^*(\omega)\left[\int_{-\infty}^{\infty} f(x)e^{-j\omega x}dx\right]d\omega\\
&= \int_{-\infty}^{\infty} F^*(\omega)F(\omega)d\omega\\
&= \int_{-\infty}^{\infty} |F(\omega)|^2 d\omega
\end{align*}

\subsection*{Transform Pairs from Table 3.2}

\subsubsection*{1. Impulse}
\textbf{Function}: $\delta(x)$
\newline
\textbf{Transform}: $\mathcal{F}\{\delta(x)\} = 1$
\newline
\textbf{Proof}:
\begin{equation*}
\mathcal{F}\{\delta(x)\} = \int_{-\infty}^{\infty} \delta(x)e^{-j\omega x}dx = e^{-j\omega \cdot 0} = 1
\end{equation*}

\subsubsection*{2. Shifted Impulse}
\textbf{Function}: $\delta(x - u)$
\newline
\textbf{Transform}: $\mathcal{F}\{\delta(x - u)\} = e^{-j\omega u}$
\newline
\textbf{Proof}: Using the shift property with $\mathcal{F}\{\delta(x)\} = 1$:
\begin{equation*}
\mathcal{F}\{\delta(x - u)\} = e^{-j\omega u} \cdot 1 = e^{-j\omega u}
\end{equation*}

\subsubsection*{3. Box Filter}
\textbf{Function}: $\text{box}(x/a) = \begin{cases} 1 & |x| \leq a \\ 0 & \text{otherwise} \end{cases}$
\newline
\textbf{Transform}: $a \cdot \text{sinc}(a\omega)$
\newline
\textbf{Proof}:
\begin{align*}
\mathcal{F}\{\text{box}(x/a)\} &= \int_{-a}^{a} e^{-j\omega x}dx\\
&= \left[\frac{e^{-j\omega x}}{-j\omega}\right]_{-a}^{a}\\
&= \frac{e^{-j\omega a} - e^{j\omega a}}{-j\omega}\\
&= \frac{2\sin(\omega a)}{\omega} \\
&= a \cdot \text{sinc}(a\omega)
\end{align*}

\subsubsection*{4. Tent Function}
\textbf{Function}: $\text{tent}(x/a) = \max(0, 1 - |x|/a)$
\newline
\textbf{Transform}: $a \cdot \text{sinc}^2(a\omega)$
\newline
\textbf{Proof}: The tent function is the convolution of two box functions:
$\text{tent}(x/a) = \text{box}(x/(a/2)) * \text{box}(x/(a/2))$

Using the convolution property:
\begin{equation*}
\mathcal{F}\{\text{tent}(x/a)\} = \mathcal{F}\{\text{box}(x/(a/2))\} \cdot \mathcal{F}\{\text{box}(x/(a/2))\} = [a\text{sinc}(a\omega/2)]^2
\end{equation*}

\subsubsection*{5. Gaussian}
\textbf{Function}: $G(x; \sigma) = \frac{1}{\sqrt{2\pi\sigma^2}} e^{-\frac{x^2}{2\sigma^2}}$
\newline
\textbf{Transform}: $\sqrt{2\pi}\sigma^{-1} G(\omega; \sigma^{-1}) = e^{-\frac{\sigma^2\omega^2}{2}}$
\newline
\textbf{Proof}: Using the integral $\int_{-\infty}^{\infty} e^{-ax^2} dx = \sqrt{\pi/a}$:
\begin{align*}
\mathcal{F}\{G(x; \sigma)\} &= \frac{1}{\sqrt{2\pi\sigma^2}} \int_{-\infty}^{\infty} e^{-\frac{x^2}{2\sigma^2}}e^{-j\omega x}dx\\
&= \frac{1}{\sqrt{2\pi\sigma^2}} e^{-\frac{\sigma^2\omega^2}{2}} \int_{-\infty}^{\infty} e^{-\frac{(x+j\sigma^2\omega)^2}{2\sigma^2}}dx\\
&= \frac{1}{\sqrt{2\pi\sigma^2}} e^{-\frac{\sigma^2\omega^2}{2}} \sqrt{2\pi\sigma^2}\\
&= e^{-\frac{\sigma^2\omega^2}{2}}
\end{align*}

\subsubsection*{6. Laplacian of Gaussian}
\textbf{Function}: $\text{LoG}(x; \sigma) = \left(\frac{x^2}{\sigma^4} - \frac{1}{\sigma^2}\right)G(x; \sigma)$
\newline
\textbf{Transform}: $-\sqrt{2\pi}\sigma^{-1}\omega^2 G(\omega; \sigma^{-1})$
\newline
\textbf{Proof}: Using the differentiation property twice:
\begin{align*}
\mathcal{F}\{G''(x; \sigma)\} &= (j\omega)^2 \mathcal{F}\{G(x; \sigma)\}\\
&= -\omega^2 e^{-\frac{\sigma^2\omega^2}{2}}\\
&= -\omega^2 \sqrt{2\pi}\sigma^{-1} G(\omega; \sigma^{-1})
\end{align*}

\subsubsection*{7. Gabor Filter}
\textbf{Function}: $\cos(\omega_0 x)G(x; \sigma)$
\newline
\textbf{Transform}: $\frac{\sqrt{2\pi}}{2}\sigma^{-1}[G(\omega - \omega_0; \sigma^{-1}) + G(\omega + \omega_0; \sigma^{-1})]$
\newline
\textbf{Proof}: Using $\cos(\omega_0 x) = \frac{1}{2}[e^{j\omega_0 x} + e^{-j\omega_0 x}]$ and frequency shifting:
\begin{align*}
\mathcal{F}\{\cos(\omega_0 x)G(x; \sigma)\} &= \frac{1}{2}[\mathcal{F}\{e^{j\omega_0 x}G(x; \sigma)\} + \mathcal{F}\{e^{-j\omega_0 x}G(x; \sigma)\}]\\
&= \frac{1}{2}[G(\omega - \omega_0; \sigma^{-1}) + G(\omega + \omega_0; \sigma^{-1})]
\end{align*}

\subsubsection*{8. Unsharp Mask}
\textbf{Function}: $(1 + \gamma)\delta(x) - \gamma G(x; \sigma)$
\newline
\textbf{Transform}: $(1 + \gamma) - \sqrt{2\pi\gamma}\sigma^{-1} G(\omega; \sigma^{-1})$
\newline
\textbf{Proof}: Using linearity and the transforms of impulse and Gaussian:
\begin{align*}
\mathcal{F}\{(1 + \gamma)\delta(x) - \gamma G(x; \sigma)\} &= (1 + \gamma) \cdot 1 - \gamma \cdot \sqrt{2\pi}\sigma^{-1} G(\omega; \sigma^{-1})\\
&= (1 + \gamma) - \sqrt{2\pi\gamma}\sigma^{-1} G(\omega; \sigma^{-1})
\end{align*}

\subsubsection*{9. Windowed Sinc}
\textbf{Function}: $\text{rcos}(x/(aW)) \cdot \text{sinc}(x/a)$
\newline
\textbf{Transform}: Approximates ideal low-pass filter as $W$ increases
\newline
The windowed sinc uses a raised cosine (Hann) window:
\begin{equation*}
\text{rcos}(x) = \frac{1}{2}(1 + \cos(\pi x))\text{box}(x)
\end{equation*}

The transform involves the convolution of the sinc transform (rectangular frequency response) with the transform of the raised cosine window, resulting in a smooth approximation to an ideal low-pass filter with reduced ringing artifacts.


\begin{thebibliography}{20}

\bibitem{szeliski2021}
R. Szeliski, \emph{Computer Vision: Algorithms and Applications}, 2nd edition, Springer, 2021.

\bibitem{szeliski2010}
R. Szeliski, \emph{Computer Vision: Algorithms and Applications}, 1st edition, Springer, 2010.

\bibitem{gonzalez2017}
R. C. Gonzalez and R. E. Woods, \emph{Digital Image Processing}, 4th edition, Pearson, 2017.

\bibitem{pratt2007}
W. K. Pratt, \emph{Digital Image Processing}, 4th edition, John Wiley \& Sons, 2007.

\bibitem{mit-fourier}
A. V. Oppenheim, R. W. Schafer, and J. R. Buck, \emph{Lecture 9: Fourier Transform Properties}, MIT OpenCourseWare, Signals and Systems, Retrieved from \url{https://ocw.mit.edu/courses/res-6-007-signals-and-systems-spring-2011/403577752d3007ed4ea30e8c61cf90d7_MITRES_6_007S11_lec09.pdf}, 2011. 

\bibitem{herman-fourier}
R. Herman, \emph{9.5: Properties of the Fourier Transform}, LibreTexts, Retrieved from \url{https://math.libretexts.org/Bookshelves/Differential_Equations/Introduction_to_Partial_Differential_Equations_(Herman)/09%3A_Transform_Techniques_in_Physics/9.05%3A_Properties_of_the_Fourier_Transform}, 2021. 
\end{thebibliography}

\end{document}
